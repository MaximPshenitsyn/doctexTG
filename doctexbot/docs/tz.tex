\documentclass[12pt]{article}
\usepackage{float}
\usepackage{tempora}
\usepackage{ragged2e}
\usepackage{multicol}
\usepackage{setspace}
\usepackage{graphicx}
\usepackage{geometry}
\usepackage{fancyhdr}
\usepackage{newtxmath}
\usepackage{afterpage}
\usepackage{indentfirst}
\usepackage[T2A]{fontenc}
\usepackage{tablefootnote}
\usepackage[utf8]{inputenc}
\usepackage[utf8]{inputenc}
\usepackage[russian]{babel}
\usepackage{hyperref}
\usepackage{systeme}
\hypersetup{
    colorlinks, citecolor=black,
    filecolor=black, linkcolor=black, urlcolor=black
}
 \geometry{
 a4paper,
 total={170mm,257mm},
 left=20mm, right=10mm,
 top=25mm, bottom=15mm,
 }

\setlength{\columnsep}{1cm}
\pagestyle{fancy}
\justifying
\graphicspath{images}

\newcommand{\header}[1]{\fancyhf{}{}\chead{#1}}
\newcommand{\footer}{\vspace*{\fill}
\begin{table}[H]
    \centering
    \begin{tabular}{|l|l|l|l|l|}
    \hline
         & & & & \\
    \hline
         Изм. & Лист & № докум. & Подп. & Дата \\    
    \hline
     RU.17701729.10.03-01 ТЗ 01-1 & & & & \\
    \hline
    Инв. № подл. & Подп. и дата & Взам. инв. № & Инв. № дубл. & Подп. и дата \\
    \hline
    \end{tabular}
\end{table}}

\begin{document}
\thispagestyle{empty}
\begin{center}{
    \textbf{
        ПРАВИТЕЛЬСТВО РОССИЙСКОЙ ФЕДЕРАЦИИ \\
        ФЕДЕРАЛЬНОЕ ГОСУДАРСТВЕННОЕ АВТОНОМНОЕ \\ 
        ОБРАЗОВАТЕЛЬНОЕ УЧРЕЖДЕНИЕ ВЫСШЕГО ОБРАЗОВАНИЯ \\
        НАЦИОНАЛЬНЫЙ ИССЛЕДОВАТЕЛЬСКИЙ УНИВЕРСИТЕТ \\
        «ВЫСШАЯ ШКОЛА ЭКОНОМИКИ»
    }
}
\end{center}
\begin{center}
Факультет компьютерных наук \\
Образовательная программа «Программная инженерия»
\end{center}

\begin{minipage}{\textwidth}{
    \begin{multicols*}{2}{
        \begin{center}{
            СОГЛАСОВАНО \\ 
            Приглашенный преподаватель ФКН ДПИ \\
            \underline{\hspace{3cm}} Иванов И. И. \\
            «\underline{\hspace{0.5cm}}»\underline{\hspace{2cm}} 2023 г.
        }\end{center}
        \columnbreak
        \begin{center}{
            УТВЕРЖДАЮ \\ 
            Академический руководитель \\ образовательной программы \\ «Программная инженерия» \\ профессор департамента программной инженерии, канд. техн. наук \\
            \underline{\hspace{3cm}} В. В. Шилов \\ 
            «\underline{\hspace{0.5cm}}»\underline{\hspace{2cm}} 2023 г.
        }\end{center}
    }
\end{multicols*}
}
\end{minipage}
\\

\hspace*{-2.0cm}
\begin{minipage}[t][3ex][l]{0.0\textwidth}
    \begin{table}[H]
    \centering
    \begin{tabular}{|c|c|}
        \hline
        \rotatebox{90}{\textbf{Подп. и дата}} &  \\
        \hline 
        \rotatebox{90}{\textbf{Инв. № дубл.}} &  \\
        \hline
        \rotatebox{90}{\textbf{Взам. инв. №}} &  \\
        \hline
        \rotatebox{90}{\textbf{Подп. и дата}} &  \\
        \hline
        \rotatebox{90}{\textbf{Инв. № подл}} &  \\
        \hline
    \end{tabular}
\end{table}
\end{minipage}


\begin{center}
    \textbf{НАЗВАНИЕ ПРОЕКТА}
\end{center}
\begin{center}
    \textbf{\fontsize{14pt}{14pt}\selectfont{Техническое задание}}
\end{center}
\begin{center}
    \textbf{\fontsize{14pt}{14pt}\selectfont{ЛИСТ УТВЕРЖДЕНИЯ}}
\end{center}
\begin{center}
    \textbf{\fontsize{14pt}{14pt}\selectfont{RU.17701729.10.03-01 ТЗ 01-1-ЛУ}}
\end{center}

\vspace*{\fill}
\begin{minipage}{\textwidth}
    \begin{multicols*}{2}{
    \begin{center}
    \end{center}
    \columnbreak
    \begin{center}{
        Исполнитель \\ 
        студент группы БПИ204 \\
        \underline{\hspace{3cm}} Э. В. Адамян \\ 
        «\underline{\hspace{0.5cm}}»\underline{\hspace{2cm}} 2022 г.
    }\end{center}
}
\end{multicols*}
\end{minipage}

\vspace*{\fill}
\begin{center}
    \textbf{\fontsize{14pt}{14pt}\selectfont{Москва 2023}}
\end{center}

\newpage
\thispagestyle{empty}
\begin{minipage}{0.5\textwidth}
\begin{center}
        УТВЕРЖДЕН \\
RU.17701729.10.03-01 ТЗ 01-1-ЛУ
\end{center}
\end{minipage}

\hspace*{-2.0cm}
\begin{minipage}[t][3ex][l]{0.0\textwidth}
    \begin{table}[H]
    \centering
    \begin{tabular}{|c|c|}
        \hline
        \rotatebox{90}{\textbf{Подп. и дата}} &  \\
        \hline 
        \rotatebox{90}{\textbf{Инв. № дубл.}} &  \\
        \hline
        \rotatebox{90}{\textbf{Взам. инв. №}} &  \\
        \hline
        \rotatebox{90}{\textbf{Подп. и дата}} &  \\
        \hline
        \rotatebox{90}{\textbf{Инв. № подл}} &  \\
        \hline
    \end{tabular}
\end{table}
\end{minipage}

\vspace*{\fill}
\begin{center}
    \textbf{НАЗВАНИЕ ПРОЕКТА}
\end{center}
\begin{center}
    \textbf{\fontsize{14pt}{14pt}\selectfont{Техническое задание}}
\end{center}
\begin{center}
    \textbf{\fontsize{14pt}{14pt}\selectfont{RU.17701729.10.03-01 ТЗ 01-1}}
\end{center}
\begin{center}
    \textbf{\fontsize{14pt}{14pt}\selectfont{Листов 14}}
\end{center}

\vspace*{\fill}
\vspace*{\fill}
\vspace*{\fill}
\begin{center}
    \textbf{\fontsize{14pt}{14pt}\selectfont{Москва 2023}}
\end{center}

\onehalfspacing
\newpage
\header{\thepage \\ RU.17701729.10.03-01 ТЗ 01-1}
\tableofcontents

\newpage
\section{Введение}
\subsection{Наименование программы}

\subsection{Наименование программы на английском}

\subsection{Краткое название программы}


\subsection{Краткая характеристика области применения программы}


\newpage
\section{Основания для разработки}
\subsection{Документы, на основании которых ведется разработка}
Программа выполняется в рамках темы курсового проекта в соответствии с учебным планом подготовки бакалавров по направлению 09.03.04 «Программная инженерия» Национального исследовательского университета «Высшая школа экономики», факультет компьютерных наук, департамент программной инженерии.

Основанием для разработки является учебный план подготовки бакалавров по направлению 09.03.04 "Программная инженерия" и утвержденная академическим руководителем тема курсового проекта.

\newpage
\section{Назначение разработки}
\subsection{Функциональное назначение}


\subsection{Эксплуатационное назначение}



\newpage
\section{Требования к программе}
\subsection{Требования к функциональным характеристикам}


\subsection{Требования к интерфейсу}

\subsection{Требования к надежности}
\subsubsection{Требования к обеспечению надежного функционала программы}
\subsubsection{Время восстановления после отказа}
\subsubsection{Отказы из-за некорректных действий оператора}

\subsection{Условия эксплуатации}
\subsubsection{Климатические условия эксплуатации}
\subsubsection{Требования к видам обслуживания}
\subsubsection{Требования к численности и квалификации персонала}

\subsection{Требования к составу и параметрам технических средств}

\subsection{Требования к информационной и программной совместимости}
\subsubsection{Требования к информационным структурам и методам решения}
\subsubsection{Требования к программным средствам, используемым программой}
\subsubsection{Требования к исходным кодам и языкам программирования}
\subsubsection{Требования к защите информации и программы}

\subsection{Требования к маркировке и упаковке}

\subsection{Требования к транспортировке и хранению}
\subsubsection{Требования к транспортировке и хранению программных документов, предоставляемых в печатном виде}

\subsection{Специальные требования}

\newpage
\section{Требования к программной документации}
\subsection{Предварительный состав программной документации}
\subsection{Специальные требования к программной документации}

\newpage
\section{Технико-экономические показатели}
\subsection{Ориентировочная экономическая эффективность}
\subsection{Предполагаемая потребность}
\subsection{Экономические преимущества разработки по сравнению с отечественными и зарубежными образцами или аналогами}

\addtocontents{toc}{\protect\newpage}
\newpage
\section{Стадии и этапы разработки}


\newpage
\section{Порядок контроля и приемки}
\subsection{Виды испытаний}
\subsection{Общие требования к приемке работы}

\newpage
\section*{Лист регистраций изменений}
\addcontentsline{toc}{section}{Лист регистраций изменений}
\begin{table}[h]
    \resizebox{\textwidth}{!}{
    \begin{tabular}{|p{1cm}|p{2.5cm}|p{2.5cm}|p{1.5cm}|p{3.5cm}|p{2cm}|p{2cm}|p{3cm}|p{1cm}|p{1cm}|} 
    \hline
    \multicolumn{10}{|c|}{\fontsize{14pt}{14pt}\selectfont{Лист регистраций изменений}}\\
    \hline
    \multicolumn{5}{|c|}{\fontsize{14pt}{14pt}\selectfont{Номера листов (страниц)}} & \multicolumn{5}{|c|}{} \\
    \hline
    \fontsize{14pt}{14pt}\selectfont{Изм.} & \fontsize{14pt}{14pt}\selectfont{Измененных} & \fontsize{14pt}{14pt}\selectfont{Замененных} & \fontsize{14pt}{14pt}\selectfont{Новых} & \fontsize{14pt}{14pt}\selectfont{Аннулированных} & \fontsize{14pt}{14pt}\selectfont{Всего листов (страниц в документе)} & \fontsize{14pt}{14pt}\selectfont{№ документа} & \fontsize{14pt}{14pt}\selectfont{Входящий № сопроводительного докум. и дата} & \fontsize{14pt}{14pt}\selectfont{Подп}. & \fontsize{14pt}{14pt}\selectfont{Дата} \\
    \hline
        \rule{0pt}{24pt}
        & & & & & & & & & \\ 
    \hline
        \rule{0pt}{24pt}
        & & & & & & & & & \\ 
    \hline
        \rule{0pt}{24pt}
        & & & & & & & & & \\ 
    \hline
        \rule{0pt}{24pt}
        & & & & & & & & & \\ 
    \hline
        \rule{0pt}{24pt}
        & & & & & & & & & \\ 
    \hline
        \rule{0pt}{24pt}
        & & & & & & & & & \\ 
    \hline
        \rule{0pt}{24pt}
        & & & & & & & & & \\ 
    \hline
        \rule{0pt}{24pt}
        & & & & & & & & & \\ 
    \hline
        \rule{0pt}{24pt}
        & & & & & & & & & \\ 
    \hline
        \rule{0pt}{24pt}
        & & & & & & & & & \\ 
    \hline
        \rule{0pt}{24pt}
        & & & & & & & & & \\ 
    \hline
        \rule{0pt}{24pt}
        & & & & & & & & & \\ 
    \hline
    \end{tabular}}
\end{table}
\end{document}
